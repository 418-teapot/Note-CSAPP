\chapter{计算机系统漫游}

计算机系统是由硬件和系统软件组成的,它们共同工作来运行应用程序。虽然系统的具体实现方式随着时间不断变化,但是系统内在的概念却没有改变。所有计算机系统都有相似的硬件和软件组成,它们又执行着相似的功能。

我们将跟踪 hello 程序的生命周期来开始对系统的学习————从它被程序员创建开始,到在系统上运行,输出简单的信息,然后中止。

hello 程序(hello.c):
\begin{cppcode}
#include <stdio.h>

int main() {
    printf("hello, world\n");
    return 0;
}
\end{cppcode}

\section{信息就是位 + 上下文}

hello 程序的生命周期是从源文件开始的,本质上是比特序列,像 hello.c 这样只由字符构成的文件称为文本文件,所有其他文件都称为二进制文件。

hello.c 的表示方法说明了一个基本思想:系统中所有的信息,包括磁盘文件、内存中的程序、内存中存放的用户数据以及网络上传送的数据,都是由一串比特表示的。区分不同数据对象的唯一方法是我们读到这些数据对象时的上下文。

\section{程序被其他程序翻译成不同的格式}

为了在系统上运行 hello.c 程序,每条 C 语句都必须被其他程序转化为一系列的低级机器语言指令。然后这些指令按照一种称为可执行目标程序(可执行目标文件)的格式打包,并以二进制文件的形式存放起来。

Linux 系统上,从源文件到目标文件的转化是由编译期驱动程序(例如 GCC 或者 clang)完成的,这个翻译过程可以分为四个阶段,如图所示。执行这四个阶段的程序(预处理器,编译期,汇编器和链接器)一起构成了编译系统(compilation system)。

\begin{tikzpicture}
    \coordinate (begin) at (0,0);
    \node[draw, inner sep=6pt, align=center] (cpp) [right=of begin, xshift=0.5cm] {Pre-\\processor\\(cpp)};
    \node[draw, inner sep=6pt, align=center] (cc1) [right=of cpp, xshift=0.5cm] {Compiler\\(cc1)};
    \node[draw, inner sep=6pt, align=center] (as)  [right=of cc1, xshift=0.5cm] {Assembler\\(as)};
    \node[draw, inner sep=6pt, align=center] (ld)  [right=of as, xshift=0.9cm] {Linker\\(ld)};
    \coordinate [right=of ld, xshift=0.9cm] (end);
    \coordinate [above=0.8cm of ld.west, xshift=-1.5cm] (tmp0);
    \coordinate [above=0.5cm of ld.west] (tmp1);
    \draw[thick, ->] (0,0) -- node[above] {$hello.c$} node[below, align=center] {Source\\program\\(text)} (cpp.west);
    \draw[thick, ->] (cpp.east) -- node[above] {$hello.i$} node[below, align=center] {Modified\\source\\program\\(text)} (cc1.west);
    \draw[thick, ->] (cc1.east) -- node[above] {$hello.s$} node[below, align=center] {Assembly\\program\\(text)} (as.west);
    \draw[thick, ->] (as.east) -- node[above] {$hello.o$} node[below, align=center] {Relocatable\\object\\programs\\(binary)} (ld.west);
    \draw[thick, ->] (ld.east) -- node[above] {$hello$} node[below, align=center] {Executable\\object\\program\\(binary)} (end.west);
    \draw[thick, ->] (tmp0) node[above] {$printf.o$} |- (tmp1);
\end{tikzpicture}

\begin{itemize}
    \item 预处理阶段。预处理器(cpp)根据以字符 \# 开头的命令,修改原始的 C 程序。比如 hello.c 中第 1 行的 \cppinline{#include <stdio.h>} 命令告诉预处理器读取系统头文件 stdio.h 的内容,并把它直接插入程序文本中。结果就得到了另一个 C 程序,通常是以 .i 作为文件扩展名。
    \item 编译阶段。编译器(cc1)将文本文件 hello.i 翻译成文本文件 hello.s ,它包含一个汇编语言程序。该程序包含函数 main 的定义,如下所示:
    \begin{gascode}
    main:
        subq $8, %rsp
        movl $.LC0, %edi
        call puts
        movl $0, %eax
        addq $8, %rsp
        ret
    \end{gascode}
    定义中 2\~{}7 行的每条语句都以一种文本格式描述了一条低级机器语言指令。汇编语言是非常有用的,因为它为不同高级语言的不同编译器提供了通用的输出语言。
    \item 汇编阶段。接下来,汇编器(as)将 hello.s 翻译成机器语言指令,把这些指令打包成一种叫做可重定位目标程序(relocatable object program)的格式,并将结果保存在目标文件 hello.o 中。
    \item 链接阶段。hello 程序调用了 printf 函数,它是 C 编译器提供的标准 C 库中的一个函数。printf 函数存在于一个名为 printf.o 的单独的预编译好了的目标文件中,而这个文件必须以某种方式合并到我们的 hello.o 程序中。链接器(ld)就负责处理这种合并。结果就得到 hello 文件,它是一个可执行目标文件,可以被加载到内存中,由系统执行。
\end{itemize}

\section{了解编译系统如何工作是大有益处的}

\begin{itemize}
    \item 优化程序性能。
    \item 理解链接时出现的错误。
    \item 避免安全漏洞。
\end{itemize}

\section{处理器读并解释储存在内存中的指令}

\subsection{系统的硬件组成}

\subsubsection{总线}

贯穿整个系统的是一组电子管道,称作总线,它携带信息字节并负责在各个部件间传递。通常总线被设计成传送定长的字节块,也就是字(word)。

\subsubsection{I/O设备}

I/O(输入/输出)设备是系统与外部世界的联系通道。每个 I/O 设备都童工一个控制器或适配器与 I/O 总线相连。

\subsubsection{主存}

主存是一个临时存储设备,在处理器执行程序时,用来存放程序和程序处理的数据。从物理上来说,主存是由一组动态随机存储器(DRAM)芯片组成的。从逻辑上来说,存储器是一个线性的字节数组,每个字节都有其唯一的地址(数组索引),这些地址是从 0 开始的。

\subsubsection{处理器}

中央处理单元(CPU),简称处理器,是解释(或执行)存储在主存中指令的引擎。处理器的核心是一个大小为一个字的存储设备(或寄存器),称为程序计数器(PC)。在任何时刻,PC 都指向主存中的某条机器语言指令(即含有该条指令的地址)。

从系统通电开始,直到系统断电,处理器一直在不断地执行程序计数器指向的指令,再更新程序计数器,使其指向下一条指令。

\subsection{运行 hello 程序}

初始时,shell 程序执行它的指令,等待我们输入一个命令。当我们在键盘上输入字符串 ``./hello'' 后,shell 程序将字符逐一读入寄存器,再把它存放到内存中。

当我们在键盘上敲回车键时,shell 程序就知道我们已经结束了命令的输入。然后 shell 执行一系列指令来加载可执行的 hello 文件,这些指令将 hello 目标文件中的代码和数据通过 DMA 的方式直接从磁盘复制到主存。数据包括最终会被输出的字符串 \cppinline{"hello, world\n"}。

一旦目标文件 hello 中的代码和数据被加载到主存,处理器就开始执行 hello 程序的 main 中的机器语言指令。这些指令将 \cppinline{"hello, world\n"} 字符串中的字节从主存复制到寄存器中,再从寄存器中复制到显示设备,最终显示在屏幕上。

\section{高速缓存至关重要}

针对处理器与主存之间的差异,系统设计者采用了更小更快的存储设备,称为 cache,作为暂时的集结区域,存放处理器近期可能会需要的信息。比较新的、处理能力更强大的系统有三级 cache:L1、L2 和 L3。系统可以获得一个很大的存储器,同时访问速度也很快,原因是利用了高速缓存的局部性原理,即程序具有访问局部区域里的数据和代码的趋势。通过让高速缓存里存放可能经常性访问的数据,大部分的内存操作都能在快速的高速缓存中完成。

\section{存储设备形成层次结构}

在每个计算机系统中的存储设备都被组织成了一个存储器层次结构,如图所示。在这个层次结构中,从上至下,设备的访问速度越来越慢、容量越来越大,并且每字节的造价也越来越便宜。

\endinput
