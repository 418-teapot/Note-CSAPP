\chapter{优化程序性能}

编写高效程序需要做到以下几点:第一,我们必须选择一组适当的算法和数据结构。第二,我们必须编写出编译器能够有效优化以转换成高效可执行代码的源代码。第三,针对处理运算量特别大的计算,将一个任务分成多个部分,这些部分可以在多核和多处理器的某种组合上并行地计算。对于第二点,理解优化编译器的能力和局限性是很重要的。

理想的情况是,编译器能够接受我们编写的任何代码,并产生尽可能高效的、具有指定行为的机器级程序。现代编译器采用了复杂的分析和优化形式,然而,即使是最好的编译器也受到 optimization blocker 的阻碍。程序员必须编写容易优化的代码,以帮助编译器。

程序优化的第一步就是消除不必要的工作,让代码尽可能有效地执行所期望的任务。这包括消除不必要的函数调用、内存条件测试和内存引用。这些优化不依赖于目标机器的任何具体属性。为了使程序性能最大化,程序员和编译器都需要一个目标机器的模型,指明如何处理指令,以及各个操作的时序特性。了解了处理器的运作,我们就可以进行程序优化的第二部,利用处理器提供的 instruction-level parallelism 能力,同时执行多条指令。

研究程序的汇编代码表示是理解编译器以及产生的代码会如何运行的最有效手段之一。仔细研究内循环的代码是一个很好的开端,识别出降低性能的属性,例如过多的内存引用和对寄存器使用不当。从汇编代码开始,我们还可以预测什么操作会并行执行,以及它们会如何使用处理器资源。常常会通过确认 critical path 来决定执行一个循环所需要的时间。

\section{优化编译器的能力和局限性}

现代编译器运用复杂精细的算法来确定一个程序中计算的是什么值,以及它们是被如何使用的,然后利用一些机会来简化表达式。

编译器必须很小心地对程序只使用安全的优化,也就是说对于程序可能遇到的所有可能的情况,优化后得到的程序和未优化的版本有一样的行为。例如:

\begin{cppcode}
void twiddle1(long *xp, long *yp) {
  *xp += *yp;
  *xp += *yp;
}

void twiddle2(long *xp, long *yp) {
  *xp += 2 * *yp;
}
\end{cppcode}

看起来,这两个过程似乎有相同的行为。不过考虑 xp 等于 yp 的情况。此时,函数 twiddle1 会执行下面的计算:

\begin{cppcode*}{autogobble=false, firstnumber=2}
  *xp += *xp;  // Double value at xp
  *xp += *xp;  // Double value at xp
\end{cppcode*}
结果是 xp 的值增加 4 倍。另一方面,函数 twiddle2 会执行下面的计算:

\begin{cppcode*}{autogobble=false, firstnumber=7}
  *xp += 2 * *xp;  // Triple value at xp
\end{cppcode*}
结果是 xp 的值增加 3 倍。编译器不知道 twiddle 会如何被调用,因此它必须假设参数 xp 和 yp 可能会相等。因此,它不能产生 twiddle2 风格的代码作为 twiddle1 的优化版本。

这种两个指针可能指向同一个内存位置的情况称为 memory aliasing。在只执行安全的优化中,编译器必须假设不同的指针可能会指向内存中同一个位置。

这造成了一个主要的 optimization blocker,这也是可能严重限制编译器产生优化代码机会的程序的一个方面。如果编译器不能确定两个指针是否指向同一个位置,就必须假设什么情况都有可能,这就限制了可能的优化策略。

\section{表示程序性能}

我们引入度量标准 Cycles Per Element,CPE,作为一种表示程序性能并指导我们改进代码的方法。CPE 这种度量标准帮助我们在更细节的级别上理解迭代程序的循环性能。

\section{程序示例}

为了说明一个抽象的程序是如何被系统地转换成更有效的代码的,我们将使用一个基于下图所示向量数据结构的运行示例。

\begin{tikzfig}
    \coordinate (A) at (0, 0);
    \coordinate (B) at (2, 0);
    \coordinate (C) at (0, -1);
    \coordinate (D) at (B |- C);
    \coordinate (ac1) at ($(A)!1/2!(C)$);
    \coordinate (bd1) at ($(B)!1/2!(D)$);
    \fill[fill=White!80!ProcessBlue] (A) rectangle (D);
    \path (A) rectangle node{len} (bd1);
\end{tikzfig}
